\section{Defect-based Test Design Technique}
\label{sec:Defect-based Test Design Technique}

\index{Defect-based}Well, this is not a technique (an approach, I suppose), but it's worthing your attentive attention.

If you will focus on searching defects, not on checking if requirements are implemented — what will you see?

When the target defects are determined based on taxonomies (a \index{Taxonomy}Taxonomy is a hierarchical list) that list root causes, defects, and failures, adequate coverage is achieved when sufficient tests are created to detect the target defects and no additional practical tests are suggested.

\textbf{Example}: list some potential defects for your app. Or make some assumptions, like 'User add several items to cart, and each entry will not disappear with each new item added' and — check it out. If this will happen for real — hooray, you just supposed a bug and you really made him discovered! 

If not — suppose something else, and check it out.

This approach is far, far away from reading basic requirements, add an item to cart, as expected, and declare, that everything is ok.