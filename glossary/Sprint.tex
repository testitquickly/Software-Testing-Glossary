\section{Sprint}
\label{sec:Sprint}

The term \rindex{\textbf{S}!Sprint}\emph{Sprint} is coming from sport. It is the act of running over a short distance at top speed.

\begin{quote}
Opposite is 'stayer'~\textemdash~a long-distance runner.\end{quote} 

Depending of physical statement, runners specialize in running on long or short distances, because this are two different strategies of running.

\begin{quote}
Sprinters runs very fast, but only at short distances (no more than 100 m). They should make an extra effort from start, immediately, without any compromises, without trying to save some power or even air breathing. Human physiology dictates that a runner's near-top speed cannot be maintained for more than 30–35 seconds due to the accumulation of lactic acid in muscles. Just get up and run!

Stayers starts slowly, at first they should try to hold on the group of all runners, at maximum saving energy. They will need all their power and speed in the latter stages of running, closer to the finish band. For example, marathon runners can run 42 km (plus ~195 m).                                                                                                                                                                                                                                                   \end{quote} 

In \rindex{\textbf{A}!Agile Software Development}Agile Software Development [p.\pageref{sec:Agile Software Development}] the term 'sprint' means almost the same~\textemdash~a short-time activity (typically one week long), where all the people involved in a project, being focused on development of some task for the project, give maximum of their power and attention.

It is obvious, that to be a real sprinter 'week by week' without long rest it is impossible. In fact, everyone involved in project use a Stayer strategy. But nobody cares, because not everyone even understand the meaning of 'Sprint' term. You want me to say 'a sprint' instead of 'a week'? No problem. \rindex{\textbf{S}!SCRUM}Scrum, sprint, scope~\textemdash~whatever. You are the boss.
