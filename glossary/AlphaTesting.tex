\section{Alpha Testing}
\label{sec:Alpha Testing}

\index{Alpha Testing}Alpha Testing is the name of an approach to testing, where the Development team ships his work to \emph{internal} testing team only.

\begin{quote}
And the \index{Beta Testing}\emph{Beta Testing} [p.\pageref{sec:Beta Testing}] means that the Development team will select a small bunch of potential users of their product (usually outside from the development company), and will ship to them the Product 'as is', just for testing purposes.

And there is no \emph{Gamma} or \emph{Delta} testing.
\end{quote}

Alpha Testing approach is specific for development Internet Shops.

Surely, technically we can always publish an Shop with some bugs, and '\textit{…if users will tell us about it, we will fix them quickly. It is just a question of \index{Cost of Failure}'cost of failure}'. 

Well, today the cost of failure looks very low, because 'You can easily fix the software online'.

\begin{quote}
\index{Skype}Skype, \index{Firefox}Firefox, \index{Twitter}Twitter, \index{Facebook}Facebook — they always ship first, then fix.
\end{quote}

But 'Shops' always means 'Money', and money requires confidence. We cannot ship an Internet Shop with any bugs in functionality, because it can scarry future customers. Or because they can cheat, and again, this is all about money. 

And not every bug can be fixed quickly.

\textbf{Main problem} — internal testing cannot reveal EVERY issue, that can happens in production. It can assure only that all functionality works as expected in expected scenarios and conditions. 

Often this limitation is very clear to development, but not for the \index{Customer}Customer.