\section{Suspension and Resumption Criteria}
\label{sec:Suspension and Resumption Criteria}
\begin{quote}
Understand what is a \rindex{\textbf{C}!Criteria}Criteria first [p.\pageref{sec:Criteria}].
\end{quote} 

\rindex{\textbf{E}!Entry criteria}\emph{Entry and Exit criteria} are the conditions which when satisfies, the test team starts (enters) the testing process and stops (exits) it.

\rindex{\textbf{S}!Suspention criteria}\emph{Suspension and Resumption criteria} are the conditions which when satisfies, the test team temporarily suspends (suspension) the testing process and resumes (resumption) it.

A test process may have to be suspended temporarily until the show stopper \rindex{\textbf{B}!Bug}bugs are resolved.

Examples:

    \begin{quote}
The project scope was agreed and approved by the client at the time indicated in the project schedule, but something goes wrong with the testing environment

    The project scope is agreed, the requirements are defined and approved by the client, but requirements was unexpectedly changed                                                                                                                                   

The credentials to the third-parties or/and external tools was available at the time indicated in the project schedule, but SUDDENLY they are not working.\end{quote} 
