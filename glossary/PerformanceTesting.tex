\section{Performance Testing}
\label{sec:Performance Testing}

Initially, \rindex{\textbf{P}!Performance testing}\emph{Performance} is an activity that a person does to entertain an audience, such as singing a song or acting in a play on a Broadway.

In technical language, \emph{Performance} is the ability of a mechanism to do an required action (or activity). It can be evaluated only as a numerical value.

\begin{quote}
\textbf{Example}: the old Cadillac Eldorado 1953 engine, we have reached a level of 100 km/h in 42 seconds, and this was possible because of 210 horsepower expected and available.

Performance can be evaluated, varied and measured in follow mode: 'Let's find out, if will be possible to reach a level of 100 km/h in 32 seconds at the old Cadillac Eldorado 1953 engine, if we will setup 240 horsepower instead of 230'.

Performance can be evaluated, varied and measured in follow mode 'Let's find out, how much time the old Cadillac Eldorado 1953 engine will work, if we will put a brick on the accelerator pedal and will have to wait'.
\end{quote} 

\subsection{Load \& Stress testing}
\label{sec:Load and Stress testing}

Performance testing simply implies \textbf{Load} and \textbf{Stress} testing. Difference between this testing types is logical and lies in the fact that different types of performance testing answer to different business questions.

\begin{quote}
\rindex{\textbf{L}!Load testing}Load testing help us to understand the behavior of the system under a specific \textbf{expected} load.

\rindex{\textbf{S}!Stress testing}Stress testing is normally used to understand the \textbf{upper limits} of capacity within the system.                                                                                             \end{quote} 

\textbf{Example}: there are an application with DB based on the MySQL. MySQL is a popular choice of database for use in web applications, an open-source relational database management system.

MySQL can easily support 200 hits per second (12 000 per minute, and 720 000 per hour, and 17 280 000 per day). How to test the performance of this web-site?

\begin{quote}
We can use JMeter for generate 100 hits per second. Is anything ok?

Let's generate 190 hits. Is anything ok?

Let's generate 210 hits. Is anything ok?

Let's generate 300 hits. Is anything ok?

Let's generate 500 hits. Is anything ok?

Let's generate 210 hits during 6 hours. Is anything ok?                                                       \end{quote} 

We have the same utils and we do the same things. But first it was simple Performance testing, then it became Load testing, then it became Stress testing, then it again became Load testing.

\subsection{Reliability Testing}
\label{sec:Reliability Testing}

\rindex{\textbf{R}!Reliability Testing}'Reliability' is a term for testing a software's ability to function, given environmental conditions, for a particular amount of time.

For example, turn on the coffee machine and use it 100 hours without any stop. Will be some problems in the software and functionality? Will be coffee available after this 100 hours? And if Yes, then what about 200 non-stop hours? Or 1000?

\subsection{Maintenance Testing}
\label{sec:Maintenance Testing}

\rindex{\textbf{M}!Maintenance Testing}Is that testing which is performed to either identify equipment problems, diagnose equipment problems or to confirm that repair measures have been effective. 

It can be performed at either the system level, the equipment level, or the component level.

Maintenance testing uses system performance requirements as the basis for identifying the appropriate components for further inspection or repair.

A good testing program will maintain a record of test results and maintenance actions taken. These data will be evaluated for trends and serve as the basis for decisions on appropriate testing frequency, need to replace or upgrade equipment and performance improvement opportunities.
